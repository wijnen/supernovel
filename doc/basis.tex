\documentclass{article}
\newcommand\code[1]{\par\noindent\begin{texttt}#1\end{texttt}\par}
\begin{document}
\noindent\textbf{\Large Lesverhaal scripts schrijven --- de basis}

\tableofcontents

\section{Inleiding}
Lesverhaal is een programma waarin verhalen kunnen worden gespeeld met vragen
ertussen. Hieronder staat uitgelegd hoe je zelf zo'n verhaal kunt maken.

Het verhaal bestaat uit plaatjes en mogelijk geluiden, en een script waarin
beschreven wordt wat er allemaal gebeurt. Dat script is een tekst
(.txt)-bestand.

De meeste regels in dat bestand zijn commando's voor de computer. Als je er
opmerkingen bij wilt zetten waar de computer niks mee moet doen, zet je een
\verb-#- als eerste teken op de regel.

\section{De achtergrond}
Meestal wil je een achtergrond in beeld zetten tijdens het verhaal. Dat kan met
het commando ``scene''. Je kunt zelf een achtergrond maken, of een van de
standaard-achtergronden kiezen.
\code{scene common/binnen.png\\
scene common/buiten.svg}

Je kan een nieuwe achtergrond instellen tijdens het verhaal. Als je dat doet,
worden alle personen weggehaald.

\section{Gesprekken}
Het belangrijkste onderdeel van het verhaal zijn pratende personen. In het
begin van het script moet je daarom de personen benoemen. Elke persoon heeft
een code die je in het script gebruikt om iets met de persoon te doen, een plek
waar de plaatjes te vinden zijn en een naam die in beeld komt als de persoon
praat. Alleen die naam mag spaties bevatten, de rest niet. Als je voor Anja de
code ``a'' gebruikt, krijg je dan:
\code{character a common/Anja.svg Anja}

\noindent Als je de characters gemaakt hebt, kan je ze in beeld zetten:
\code{\# Zet Anja in beeld, in het midden.\\
show a\\
\# Zet Anja in beeld aan de linkerkant.\\
show a at left\\
\# Zet Anja in beeld aan de rechterkant.\\
show a at right\\
\# Haal Anja weer weg.\\
hide a}

\noindent Om iemand te laten praten, gebruik je de code en een dubbele punt, en
dan de tekst:
\code{a: Hallo, welkom bij dit verhaal!}

\section{Vragen}
Je kan gewoon een verhaal vertellen, maar het is meestal leuker om af en toe
een vraag te stellen die de speler moet beantwoorden. Er zijn een heleboel
soorten vragen, hier worden er twee beschreven.

De vraag zelf stel je altijd op dezelfde manier: door een stuk tekst tussen
aanhalingstekens. Dat mogen zowel dubbele als enkele aanhalingstekens zijn, als
je maar dezelfde voor en na de vraag gebruikt. Als je een lange vraag hebt, wil
je die meestal op meerdere regels intypen. Dat kan door te beginnen en eindigen
met een regel met daarop drie aanhalingsteken (enkel of dubbel) en verder niks.
Dit zijn wat voorbeelden:
\code{'Wat is 1+1?'\\
\\
"Wat is 1-1?"\\
\\
'''\\
Anja stapt op maandag in de bus. Ze heeft een groene jas aan.\\
\\
Hoe heet de buschauffeur?\\
'''}

Je kan opmaak gebruiken in de vraag (en trouwens ook in de gesproken tekst).
Als je wilt weten hoe dat werkt, zoek dan op internet naar \verb-markdown-.

Nadat je de vraag gesteld hebt, wil je er natuurlijk ook antwoord op krijgen.
Elke vraag moet een code hebben. Net als bij de personen ziet de speler daar
niks van; de codes zitten alleen in het script. Je mag elke code maar een keer
gebruiken.

Om een meerkeuzevraag te maken, gebruik je het commando \verb-choice-. Op de
regels daarna moet je de mogelijke antwoorden opgeven, met het commando
\verb-option-. Bijvoorbeeld:
\code{choice richting\\
option rechts\\
option links}

Hier is de code \verb-richting-. Er zijn twee keuzes.

Over het algemeen zal je het verhaal willen aanpassen aan het antwoord van de
speler. Dat kan met het commando \verb-goto-. Je kan daarmee direct verder gaan
naar een ander stuk van het script. De plek waar je heen wilt springen moet je
een label geven. Bijvoorbeeld:
\code{a: Ik ga nu een stukje overslaan.\\
goto verder\\
a: Dit ga ik dus nooit zeggen.\\
label verder\\
a: Dit ga ik wel zeggen.}

Het overslaan van een stuk script is natuurlijk niet zo interessant; dan kan je
het beter niet in je script zetten. Je hebt er pas wat aan als je het
combineert met \verb-if-.

De code van de vraag (\verb-richting- in het voorbeeld hierboven) is het nummer
van het gekozen antwoord. De nummers beginnen bij 0. (Dus hierboven is rechts 0
en links is 1.) Met \verb-if- kan je een \verb-goto- doen, alleen als een
bepaald antwoord gegeven is. (Je kan \verb-if- ook met andere commando's
gebruiken.)
\code{'Waar wil je heen?'\\
choice richting\\
option rechts\\
option links\\
\\
if richting == 0:\\
  goto rechts\\
\\
a: Je bent naar links gegaan!\\
goto eind\\
\\
label rechts\\
a: Je bent naar rechts gegaan!\\
\\
label eind\\
a: Dit ga ik altijd zeggen.}

De andere soort vraag is er een waar een getal als antwoord op gegeven moet worden.
Die heet \verb-number-. De waarde daarvan is het getal dat is ingetypt. Je kan
meestal het best een gebied aangeven om te controleren of het antwoord goed is:
\code{'Wat is 1/3?'\\
number eenderde\\
if 0.332 < eenderde < 0.334:\\
  a: Dat klopt!\\
else:\\
  a: Helaas, dat klopt niet.}

Je ziet hier ook het commando \verb-else-. Dat kan je gebruiken om iets te doen
als de \verb-if- juist niet wordt gedaan. Let op: zowel bij \verb-if- als bij
\verb-else- moeten de commando's erna ingesprongen zijn.

Je ziet ook dat kommagetallen in Engelse notatie moeten worden geschreven, dus
met een punt in plaats van een komma.

\section{Geluid}
Je kan muziek en geluidseffecten afspelen tijdens het verhaal. Je gebruikt
daarvoor de commando's \verb-music- en \verb-sound-. Als je muziek afspeelt,
begint die vanzelf weer opnieuw aan het eind. Als je geen bestandsnaam opgeeft,
stopt de muziek juist met afspelen.
\code{music elvis.mp3\\
a: Oh leuk, ik houd van Elvis Presley!\\
sound kreun.mp3\\
a: Maar nu heb ik er wel weer genoeg van...\\
a: Ik zet hem weer uit.\\
music\\
a: Zo, die stilte is ook lekker.}

\section{Beweging}
Wanneer characters ergens heen gaan, kan dat met een beweging. Anders staan ze
plotseling op hun plaats. Als je een character op het scherm zet, kan je
daarbij \verb-with moveinleft- of \verb-with moveinright- gebruiken om het een
beweging vanaf links of rechts te maken. Als je juist van het scherm af wilt
bewegen, gebruik je \verb-with moveoutleft- of \verb-with moveoutright-. Bij
een beweging van de ene naar de andere plaats gebruik je \verb-with move-.
\code{show a at left with moveinright\\
show a at right with move\\
hide a with moveoutleft}
\end{document}
